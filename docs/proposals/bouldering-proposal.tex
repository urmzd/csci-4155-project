\documentclass[10pt]{article}
\usepackage{cite}
\usepackage{lipsum}

\title{Evaluating Bouldering Route Difficulty}
\author{ Urmzd
  Mukhammadnaim$^1$\\ \texttt{B00800045} \\ \texttt{urmzd@dal.ca} \\ \\ \\ 
  \and
  Keelin Sekerka-Bajbus$^1$\\ \texttt{B00739421}\\ \texttt{kl967083@dal.ca}\\
  \\ \\ 
  \and Benjamin J. Macdonald$^1$\\ \texttt{B00803015}\\
\texttt{bn282348@dal.ca}\\ \\ \\ } 

\bibliographystyle{apalike}

\date{$^1$Faculty of Computer Science, Dalhousie University}
\setcounter{secnumdepth}{0}

\begin{document} \maketitle

\section{Problem Statement}
In rockclimbing, the difficulty of climbing routes
is classified using a grade given by climbers, generally depending on the
length or style of route, the challenge of the required moves, and how
dangerous climbing conditions are. As a result, climbing grades are very
subjective as they depend on a collective human assessment and the many
existing grade scales are rather ambiguous when it comes to distinguihsing
between closely related difficulty classes. This presents a challenging task in
attempting to accurately predict the assignment of climbing grades using
machine learning, with the potential to benefit the performance of climbers
(particularly when training) and to aid in improving the ambiguity of
standardized scales.

With this project, we would explore predicting the climbing grade of Bouldering
problems, climbing routes that are for free climbing without ropes or harnesses
that are generally no higher than 6 metres, using the Hueco scale (V0 - V17).

\section{Data Collection and Data Processing}

To explore this challenging task, we would use climbing route data collected
from https://www.moonboard.com/. Moonboards are a standardized climbing wall
made up of 142 rock holds on a 18x11 grid that are used for indoor bouldering
training. Moonboard climbers utilize an app to load a problem route to the
board, the sequence of holds marked by illuminated LEDs, so a large dataset of
problems has been created by Moonboard and community users (over 30k problems
were scraped by Duh \& Chang in their 2021 work using RNN models for
classication and route generation).

The Moonboard route can be effectively encoded as a ${0, 1}^(18 \times 11)$ matrix to
serve as a graphical representation of the board, which can then be one-hot
encoded (or multi-hot) to prepare the representation for input to the neural
network. Essentially, the encoded climbing routes will allow the neural network
to look for patterns and simliarities between climbing grade classes and learn
to distinguish problems appropriately, particularly since the climbing routes
will follow a standard format as described above.

\section{Approaches}
We would explore this classification task by experimenting with a more advanced
Convolution Neural Network architecture (potentially) than in previous works,
and apply this architecture to the Moonboard dataset for training and
testing.The Moonboard dataset will first need to be scraped and prepared with
basic cleaning and pre-processing (e.g. standardizing the grade scale to use
Hueco instead of a European scale for instance). Data will need to undergo a
hot-encoding process or similar to prepare to be inputted to the neural
networks.The Neural Network architecture will be designed and implemented to be
applied to the climbing route data, with experiments addressing different
activation functions, learning rates, and other parameters or normalization
will be conducted to illustrate the network's performance and fine-tuning
processes. Following experimental results, we will revisit our experiment
design (or network architectures) as necessary with time permitting to
investigate the performance more closely (e.g. look to what sorts of climbing
routes were often misclassified, impact of learning rates, etc.). Results from
our experiments will be discussed thoroughly in our final report, with special
attention to illustrating the theoretical understanding of the problem and the
performance of our models.

\section{Related Works}
\subsection{Dobles et al. (2017)}
Dobles et al. (2017) employed and evaluated Naives Bayes, softmax regression,
and Convolutional Neural Network classifiers to attempt to determine the
difficulty grade of climbing routes, specifically using a dataset collected
from Moonboard.com to standardize the data. They yielded the following
results for each classifier respectively, 34.0\% 36.5\% 34.0\% \cite{dobles_sarmiento_satterthwaite_2017}.

\subsection{Duh \& Chang (2021)}
Duh \& Chang (2021) employed RNN architectures to explore classifying
Moonboard climbing route grades and to generate new Moonbaord routes. Their
'GradeNet' architecture achieved 46.7\% accuracy upon testing \cite{DBLP:journals/corr/abs-2102-01788}.

\subsection{Tai et al. (2020)}
Tai et al. (2020) applied Graph Convolutional Neural Networks (GCN)
architectures previously used in NLP applications to classifying the climbing
route grade of Moonboard problem sets, with their top model achieving an
average AUC score of 0.73 across all classes \cite{tai_wu_hinojosa_2020}.

\bibliography{bouldering-proposal}

\end{document}
