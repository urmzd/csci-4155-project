\documentclass[10pt]{article}
\usepackage{cite}
\usepackage{lipsum}
\usepackage{hyperref}

\title{Evaluating Bouldering Route Difficulty}
\author{ Urmzd
  Mukhammadnaim$^1$\\ \texttt{B00800045} \\ \texttt{urmzd@dal.ca} \\ \\ \\ 
  \and
  Keelin Sekerka-Bajbus$^1$\\ \texttt{B00739421}\\ \texttt{kl967083@dal.ca}\\
  \\ \\ 
  \and Benjamin J. Macdonald$^1$\\ \texttt{B00803015}\\
\texttt{bn282348@dal.ca}\\ \\ \\ } 

\bibliographystyle{apalike}

\date{$^1$Faculty of Computer Science, Dalhousie University}
\setcounter{secnumdepth}{0}

\begin{document} \maketitle

\section{Problem Statement}
In rock climbing, the difficulty of climbing routes
are classified using a grade given by climbers. Generally, the grades depend on the
length or style of the route, the challenge of the required moves, and the
estimated danger of the climbing conditions. As a result, climbing grades are
subjective as they depend on collective human assessment. It doesn't help that many 
existing grade scales are also ambiguous when it comes to 
distinguishing between approximate difficulty levels. 
The lack of object route assessments presents an opportunity to build a model which can accurately predict the 
the difficulty of a route, with the potential 
to benefit climbers' performance, particularly when training) and aid in
improving the ambiguity of standardized scales.

With this project, we would explore predicting the climbing grade of Bouldering
problems; routes that are designed for free climbing and do not involve the use 
of ropes or a harnesses. Additionally the routes are typically no higher than 6 metres.
For this project, we propose the use of the Hueco scale (V0 - V17).

\section{Data Collection and Data Processing}

To explore this task, we would use climbing route data collected from 
\href{https://www.moonboard.com/}{moonboard.com}. Moonboards are a standardized climbing wall
made up of 142 rock holds on an 18x11 grid. They are most often used for indoor Bouldering
training. Moonboard climbers utilize an app to load a problem route to the
board, the sequence of holds marked by illuminated LEDs, so a large dataset of
problems have already been created by Moonboard and community users. For reference, over 30k problems
were scraped by Duh \& Chang in their 2021 work using RNN models for
classification and route generation \cite{DBLP:journals/corr/abs-2102-01788}.

The Moonboard route can be effectively encoded as a ${0, 1}^(18 \times 11)$ matrix to
serve as a graphical representation of the board, which can then be one-hot
encoded (or multi-hot) to prepare the representation for input to the classification 
model. Essentially, the encoded climbing routes will allow the model 
to look for patterns and similarities between climbing grade classes and learn
to distinguish problems, since the climbing routes
will follow a standard format as described above.

\section{Approaches}
We would explore this classification task by experimenting with a more advanced
convolution neural network architecture than in previous works,
and apply this architecture to the Moonboard dataset for training and
testing. The Moonboard dataset will first need to be scraped and prepared with
basic cleaning and pre-processing (e.g. standardizing the grade scale to use
Hueco instead of a European scale). Data will need to undergo a
hot-encoding process or in preparation for its input to a neural
network. The neural network architecture will be designed and implemented to be
applied to the climbing route data, with experiments addressing different
activation functions, learning rates, and other parameters 
will be conducted to illustrate the network's performance and fine-tuning
processes. Following experimental results, we will revisit our experiment
design (or network architectures) as necessary with time permitting to
investigate the performance more closely (e.g. look to what sorts of climbing
routes were often misclassified, the impact of learning rates, etc.). Results from
our experiments will be discussed thoroughly in our final report, with special
attention to illustrating the theoretical understanding of the problem and the
performance of our models.

\section{Related Works}
\subsection{CNN for Climbing Route Classification}
Dobles et al. (2017) employed and evaluated Naives Bayes, softmax regression,
and Convolutional Neural Network classifiers to attempt to determine the
difficulty grade of climbing routes, specifically using a dataset collected
from Moonboard.com to standardize the data. They yielded the following
accuracies for each classifier respectively, 34.0\% 36.5\% 34.0\% \cite{dobles_sarmiento_satterthwaite_2017}.

\subsection{RNN for Route Classification and Generation}
Duh \& Chang (2021) employed RNN architectures to explore classifying
Moonboard climbing route grades and to generate new Moonbaord routes. Their
'GradeNet' architecture achieved 46.7\% accuracy upon testing \cite{DBLP:journals/corr/abs-2102-01788}.

\subsection{Graph Convolutional Neural Networks for Route Classification}
Tai et al. (2020) applied Graph Convolutional Neural Networks (GCN)
architectures previously used in NLP applications to classifying the climbing
route grade of Moonboard problem sets, with their top model achieving an
average AUC score of 0.73 across all classes \cite{tai_wu_hinojosa_2020}.

\bibliography{bouldering-proposal}

\end{document}
